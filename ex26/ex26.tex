\documentclass{article}
\usepackage[english]{babel}
\usepackage[letterpaper,top=2cm,bottom=2cm,left=3cm,right=3cm,marginparwidth=1.75cm]{geometry}
\usepackage{amsmath}
\usepackage{amssymb}
\usepackage{graphicx}
\usepackage[colorlinks=true, allcolors=blue]{hyperref}
\usepackage{polski}

\title{Algorytmiczna teoria gier {-} zadanie 26}
\author{Gabriel Budziński}

\begin{document}
\maketitle

\section{Treść}

Trzy małe swinki mieszkały w trzech własnoręcznie wybudowanych budynkach. Każde dwie mieszkały w odległości $\sqrt{3}$km od siebie. Chciały zamontowac alarm przeciw-wilkowy. Mają jedną czujkę, która ma zasięg wykrywania wilków do 2km. Każda świnka musi kupic kabel (niezależnie od siebie), który połączy jej domek czujką. Kilometr kabla kosztuje 1. Jesli wszystkie świnki kupią co najmniej 1km kabla, to umieszczają czujkę idealnie pośrodku trójkąta utworzonego przez domy świnek. Jeśli która świnka kupi mniejszą długo śc kabla, to czujkę stawiamy możliwie jak najbliżej centrum trójkąta. Jeśli kabli nie wystarczy, aby połączyc wszystkie 3 swinki z alarmem, to korzystają dwie świnki, które kupiły najdłuzsze kable (wszelkie remisy rozstrzygane są w kolejno ści jako ści domków, tj. wygrywa słomiany, potem drewiany i na końcu murowany). W przypadku, gdy mozna połączyc tylko dwie świnki z czujką, czujka jest stawiana możliwie najbliżej środka odcinka łączącego domy tych dwóch świnek. Jeśli dwie świnki nie mają w sumie kabli o łącznej długości 3km, to świnka, która kupiła najdłuższy kabel, stawia czujkę bezpo średnio przy swoim domu (remisy rozstrzygamy jak wcześniej). Niechroniona (niepodłączona do czujki) świnka musi zapłacić 10 za usługi firmy ochroniarskiej. Kazda podłączona do czujki świnka zyskuje trzykrotność minimalnej odległości między jej domem a brzegiem obszaru chronionego przez czujkę.

\renewcommand{\labelenumi}{\alph{enumi})}
\begin{enumerate}
    \item Dlaczego można uznać, że gra ma ograniczone wypłaty?
    \item Opisz jakie strategie czyste świnek są zdominowane.
    \item Znajdź optima Pareta oraz czyste równowagi Nasha.
    \item Czy istnieje rozwiązanie dominujące dla tej gry?
    \item Jak należałoby zmienić warunki gry, aby móc uznać tę grę za dyskooperatywną?
\end{enumerate}

\section{Rozwiązanie}

\subsection*{a)}

Wypłaty są w oczywisty sposób ograniczone z góry przez 6, ponieważ maksymalna odległość od krawędzi koła o promieniu 2km znajdując się w nim to 2km.

Z treści zadania wprost wynika, że nie opłaca się kupować kabla dłuższego niż 1km, ponieważ niezależnie od wyborów pozostałych świnek uzyska się dostęp do czujki, a kupując dłuższy kabel nie polepszy się swojej wypłaty (można tylko pogorszyć, bo większy koszt kabla i być może dalej od czujki). W takim razie możemy ostro ograniczyć wypłaty od dołu przez $-10-1 = -11$.

\subsection*{b)}

Jak powiedzieliśmy, wszelkie strategie o długości kabla większej niż 1 są silnie zdominowane, ponieważ strategia 1 zawsze da lepszą wypłatę, niezależnie od strategii pozostałych świnek. Pozostałe strategie są nieokreślone, ponieważ możemy dostać wyższą wypłatę niż ze strategią 1 (np. nasza to 0.5 a pozostałych 0), ale możemy też dostać niższą (np. my 0.5 a reszta 1).

\subsection*{c)}

\begin{itemize}
    \item Równowagi Nasha: Jedyną równowagą jest profil strategii $(1,1,1)$, ponieważ nie mając dostępu do czujki mamy maksymalnie wypłatę -10, a podłączając nawet na maksymalną odległość (czujka leży na przeciwległym boku) wypłata wynosi $3 \cdot \frac{1}{2} - \frac{3}{2} = 0$ (oczywiście lepiej jest kupić krótszy kabel długości 1, bo czujka będzie bliżej nas, wtedy najlepsza wypłata zakładając, że pozostałe świnki zawsze chcą dla nas jak najgorzej). Każde inne ustawienie niż $(1,1,1)$ nie jest najlepszą odpowiedzią, bo pozwala nam poprawić którąś z wypłat zmieniając strategię, czyli nie są równowagami Nasha.
    \item Optima Pareto: Mamy nieskończoną liczbę optimów w sensie Pareto, ponieważ każdy profil strategii, który powoduje podłączenie wszystkich świnek do czujki oraz kable są optymalnej długości (kable akurat starczają do czujki). Jeśli chcielibyśmy sobie polepszyć wypłatę, to jedynie możemy to zrobić skracając kabel, co z poprzednich rozważań z pewnością odłączyłoby którąś ze świnek i zmiejszyło jej wypłatę. Podobnie w konfiguracjach, w których nie wszystkie świnki są podłączone można polepszyć wypłatę świnek bez dostępu do czujki nie zmniejszając wypłaty tych podłączonych (podobnie jak w równowagach Nasha). Jeśli jest mniej świnke podłączonych to zawsze możemy kupić sobie kabel długości naszej odległości od czujki i się podłączyć poprawiając sobie i nie pogarszając innym.
\end{itemize}

\subsection*{d)}

Nie, ponieważ da się uzyskać większe wypłaty niż w profilu strategii (1,1,1) (np. równe 6 lub bliskie 6, w zależności, która świnka), ale są wzajemnie wykluczające.

\subsection*{e)}

Redukujemy do dwóch świnek, a następnie każemy świnkom wybrać wartość zmiennej boolowskiej, jeśli wskażą taką samą, to czujkę dostaje tylko świnka A, w p.p. tylko świnka B.

Chyba, że chodziło o antykooperacyjną, wtedy można by było np zmniejszać zasięg czujki proporcjonalnie do liczby podłączonych świnek, wtedy wypłata spadałaby wraz z liczbą graczy obierających tę samą strategię.

\end{document}