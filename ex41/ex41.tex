\documentclass{article}
\usepackage[english]{babel}
\usepackage[letterpaper,top=2cm,bottom=2cm,left=3cm,right=3cm,marginparwidth=1.75cm]{geometry}
\usepackage{amsmath}
\usepackage{amssymb}
\usepackage{graphicx}
\usepackage[colorlinks=true, allcolors=blue]{hyperref}
\usepackage{polski}
\usepackage{float}
\usepackage{slashbox}

\title{Algorytmiczna teoria gier {-} zadanie 41}
\author{Gabriel Budziński}

\begin{document}
\maketitle

\subsubsection*{(a)}

Weźmy dwa zbiory zwarte $X$ i $Y$ oraz ciągi w tych zbiorach, odpowiednio $(x_n)$ i $(y_n)$. Wiemy, że $(x_n, y_n)$ jest zbieżny do $(x, y)$ wtedy i tylko wtedy, gdy $x_n \rightarrow x$ i $y_n \rightarrow y$. Zatem jeśli $(x_n)$ mapodciąg zbieżny $(x_{n_k})$ i $(y_{n_k})$ ma podciąg zbieżny $(y_{n_{k_l}})$, to $(x_{n_{k_l}}, y_{n_{k_l}})$ jest zbieżnym podciągiem ciągu $(x_n, y_n)$. Rozszeżenie tego wniosku na produkt zkończonej liczby zbiorów zwartych jest jasne.

\subsubsection*{(b)}

Weźmy zbiory wypukłe $A$ i $B$ oraz dowolne $x_A, y_A \in A$ oraz $x_B, y_B \in B$. Oznaczmy $\{\lambda x + (1 - \lambda)y : \lambda \in [0,1]\}$ jako $[x,y]$. Wiemy, że $[x_A, y_A] \subset A$ oraz $[x_B, y_B] \subset B$, w takim razie $([x_A, y_A], [x_B, y_B]) \subset A \times B$. Z tego mamy $A \times B$ - wypukły. Rozszeżenie tego wniosku na produkt zkończonej liczby zbiorów zwartych jest jasne.

\subsubsection*{(c)}

W przestrzeni dyskretnej każdy zbiór jest otwarty i domknięty, więc każdy przeciwobraz zbioru $B \subset I$, zbiór $A = f^{-1}[B] = {x \in S: f(x) = B}$ jest zbiorem otwartym.

\subsubsection*{(d)}

Natarczywy adorator:

\begin{table}[H]
    \begin{tabular}{c||c|c|c}
        \backslashbox{$s_k$}{$s_a$} & S & G & \\\hline\hline
        S & -1 & 1 & p\\\hline
        G & 1 & -1 & 1-p\\\hline
        & q & 1-q &
    \end{tabular}
\end{table}

Gra ta nie ma czystych równowa Nasha (bo się kręcimy w kółko próbując je znaleźć, nie wiem jak dodać strzałki w tablicy).

Policzmy dla $s_k$:

$(-1)q + 1(1-q) = 1q + (-1)(1-q)$

$1-2q = 2q-1$

$q=\frac{1}{2}$

Analogicznie dla $s_a$.

Dostajemy MNE $((\frac{1}{2}, \frac{1}{2}),(\frac{1}{2}, \frac{1}{2}))$.

\subsubsection*{(e)}

Spełniają one warunek wklęsłości przez równość (są jednocześnie wklęsłe i wypukłe).

\subsubsection*{(f)}

Z twierdzenia Nasha z wykładu (4.1) mamy, że $F(s)$ jest domknięty.

\subsubsection*{(g)}

Ponieważ używając metody z uprzedniego dowodu daje kręcenie się w kółko.

\end{document}