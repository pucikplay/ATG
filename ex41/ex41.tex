\documentclass{article}
\usepackage[english]{babel}
\usepackage[letterpaper,top=2cm,bottom=2cm,left=3cm,right=3cm,marginparwidth=1.75cm]{geometry}
\usepackage{amsmath}
\usepackage{amssymb}
\usepackage{graphicx}
\usepackage[colorlinks=true, allcolors=blue]{hyperref}
\usepackage{polski}

\title{Algorytmiczna teoria gier {-} zadanie 41}
\author{Gabriel Budziński}

\begin{document}
\maketitle

\section*{(a)}

Weźmy dwa zbiory zwarte $X$ i $Y$ oraz ciągi w tych zbiorach, odpowiednio $(x_n)$ i $(y_n)$. Wiemy, że $(x_n, y_n)$ jest zbieżny do $(x, y)$ wtedy i tylko wtedy, gdy $x_n \rightarrow x$ i $y_n \rightarrow y$. Zatem jeśli $(x_n)$ mapodciąg zbieżny $(x_{n_k})$ i $(y_{n_k})$ ma podciąg zbieżny $(y_{n_{k_l}})$, to $(x_{n_{k_l}}, y_{n_{k_l}})$ jest zbieżnym podciągiem ciągu $(x_n, y_n)$. Rozszeżenie tego wniosku na produkt zkończonej liczby zbiorów zwartych jest jasne.

\section*{(b)}


\section*{(c)}


\section*{(d)}


\section*{(e)}


\section*{(f)}


\section*{(g)}



\end{document}